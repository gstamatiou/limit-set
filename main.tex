\documentclass{report}

% Packages for math symbols and equations
\input{packages}
\input{commands}

\usepackage{tikz-cd}


% Title page information
\title{Limit sets of Anosov representations}
\author{Giorgos}
\date{\today}

\begin{document}

\maketitle

\tableofcontents

\chapter{Introduction}
\begin{definition}
    Let $\Gamma$ be a discrete group of isometries of a metric space $(X,d)$.
    We define the critical exponent of $\Gamma$ to be the asymptotic exponential growth of its orbits, i.e.\ the following limit:
    \[
    \delta_\Gamma = \limsup_{n\to \infty} \frac{\log \sharp \{ \gamma \in \Gamma : d(x, \gamma x) \leq n \}}{n}
    \]
    for some fixed $x \in X$.
\end{definition}
\remark
It is not hard to show that the critical exponent is independent of the choice of $x$.
\section{Lie group preliminaries}
We fix the Cartan subalgebra $\mathfrak a$ of $\SL(d,\mathbb R)$:
\[
    \mathfrak a = \left\{
        \diag(\alpha_1, \ldots, \alpha_d) : \alpha_1 + \cdots \alpha_d = 0
    \right\}
\]
and the Weyl chamber $\mathfrak a^+$ of $\SL(d,\mathbb R)$
\[
    \mathfrak a^+ = \left\{
        \diag(\alpha_1, \ldots, \alpha_d) : \alpha_1 \geq \cdots \geq \alpha_d
    \right\}.
\]
Denoting with $K = \SO(d, \mathbb R), A^+ = e^{\mathfrak a^+} $, we have the Cartan decomposition:
\begin{align*}
    \mathfrak{sl}(d, \mathbb R) &\rightarrow K \times A^+ \times K\\
    g &\mapsto (k_g, a_g, l_g) 
\end{align*}
such that $g = k_g a_g l_g$.
In particular $a_g = \diag(\sigma_1(g), \ldots, \sigma_d(g))$ with $\sigma_1 \geq \cdots \geq \sigma_d(g)$,
where $\sigma_i(g)$ is the $i$-th singular value of $g$, i.e.\ eigenvalue of $g^t \cdot g$.

We will use the spaces
\[
    U_p(g) = \mathbb R u_1(g) \oplus \cdots \oplus \mathbb R u_p(g)   
\]
where $u_i(g) = k_g \cdot e_i$.
One can easily show that the decomposition
\[
    g^{-1} \cdot U_p(g) \oplus U_{d-p}(g^{-1})
\]
is orthogonal with respect to the standard inner product and that
$u_p(g^{-1}) = l_g^{-1} e_{d-p+1}(g).$
\section{Limit set preliminaries}
\begin{definition}
For $p \in \{2, \ldots, d\}, s\in \mathbb R $
and $g \in SL(d, \mathbb R)$ 
we denote with $\tilde \Psi_s^p(g), \Psi_s^p(g): \SL(d, \mathbb R) \to \mathbb R$ the functional:
\begin{align*}
\Psi_s^p(g) &= 
    \alpha_{12}(a(g)) + \cdots + \alpha_{1(p-1)}(a(g)) + (s - (p-2))\alpha_{1p}(a(g))\\
\tilde \Psi_s^p(g) &= 
    \left( \frac{\sigma_2}{\sigma_1}\cdots\frac{\sigma_{p-1}}{\sigma_1}(g)\right) 
    \left( \frac{\sigma_{p-1}}{\sigma_1}(g) \right)^{s - (p-2)}
\end{align*}
\end{definition}
\begin{remark}
    We have $\alpha_{ij}(a) = a_i - a_j, a_i(g) = \log (\sigma_i(g))$ and 
    \[
        \Psi_s^p(g) = \log \tilde \Psi_s^p(g)
    \]
and that
\begin{align*}
    \min_{p \in \llbracket 2, d \rrbracket} 
    \left\{ 
        \sum_{|\gamma| = T} 
            \frac{\sigma_2}{\sigma_1}\cdots\frac{\sigma_{p-1}}{\sigma_1}(g) 
            \left( \frac{\sigma_{p-1}}{\sigma_1}(g) \right)^{s - (p-2)}
    \right\} = 
    \sum_{|\gamma| = T} e^{-\max\limits_{p \in \llbracket 2, d \rrbracket} \Psi_s^p(g)}
\end{align*}
\end{remark}

\begin{remark}
For any $g \in \SL(d, \mathbb R)$ we have that:
\[
    \max_{p \in \llbracket 2, d \rrbracket} \Psi_s^p(g) = \Psi_s^{p_0}(g) \text{ for } s \in [p_0 - 2, p_0 -1].
\]
Indeed, a quick calculation shows that for $s \geq 0$ and $p \in \llbracket 2, d \rrbracket$:
\[
    \Psi_s^p(g) \leq \Psi_s^p(g) \text{ if and only if } s \geq p-1.
\]
and that equality holds in the case $s = p - 1$.
Thus for $s \in [p - 2, p-1]$ we have that
\begin{align*}
    s \geq p-2, \ldots, 1 &\text{ implies that } \Psi_s^p(g) \geq \ldots \geq \Psi_s^{2}(g)\\
    s \leq p, \ldots, d-1 &\text{ implies that } \Psi_s^p(g) \leq \ldots \leq \Psi_s^d(g)
\end{align*}
Another way to see this (refer to \cref{fig:max}) is to note that $\Psi_s^2(g), \cdots, \Psi_s^d(g)$ is a sequence of functions that are affine in $s$, with slopes $\alpha_{12}(g) \leq \cdots \leq \alpha_{1d}(g)$ and that they satisfy $\Psi_1^2(g) = \Psi_2^2(g), \Psi_2^3(g) = \Psi_3^4(g) \ldots, \Psi_{d-2}^{d-1}(g) = \Psi_{d-2}^d(g)$.
\begin{figure}[h]
    \centering
    \includegraphics[width=0.4\textwidth]{max.jpg}
    \caption{Visual illustration that $\max_p\Psi_s^p(g) = \Psi_s^{p_0}(g) \text{ for } s \in [p_0 - 2, p_0 -1]$.}
    \label{fig:max}
\end{figure}    
\end{remark}


The following definition comes from \cite{ledrappier_dimension_2023}, in the special case of projective Anosov representations ($P = {1}$):
\begin{definition}
    For $s \geq 0$ we consider the Falconer functional $F_s : \SL(d, \mathbb R) \to \mathbb R$ by:
    \[
        F_s(g) = \min 
        \left\{
            \sum_{j=2}^d s_j \alpha_{1j}(a(g)) : s_j \in (0,1], \sum_{j=2}^d s_j = s 
        \right\},
    \]
    and define the Falconer dimension $\dim_F(\rho)$ of $\rho$ to be its critical exponent:
    \[
        \dim_F(\rho) = \inf
        \left\{
            s > 0: \sum_{\gamma \in \Gamma} e^{-F_s(\rho(\gamma))} < \infty
        \right\}.
    \]
\end{definition}

\begin{remark}
    Using elementary computations one may prove that for all $s \geq 0$:
    \[
        F_s(g) = \max_{p \in \llbracket 2, d \rrbracket} \Psi_s^p(g)
    \]
\end{remark}

\begin{definition}
Let $\rho: \Gamma \to \SL(d, \mathbb R)$ be a linear representation and $p \in \llbracket 1, d-1 \rrbracket$.
We say that $\rho$ is $p$-Anosov if there exist constants $\mu, C>0$ such that for all $\gamma \in \Gamma$:
\[
    \frac{\sigma_{p+1}}{\sigma_p}(\rho(\gamma)) \leq C e^{-\mu |\gamma|}.
\]
One can show that in that case there exist equivariant continuous maps $\xi^p: \hat \Gamma \to \mathcal G_p(\mathbb R^d) , \xi^{d-p}: \hat \Gamma \to \mathcal G_{d-p}(\mathbb R^d)$ that are transverse and restrict to
\[
    \xi^p(\gamma) = U_p(\rho(\gamma)), \xi^{d-p}(\gamma) = U_{d-p}(\rho(\gamma))
\]
for $\gamma \in \Gamma$, where $U_p(\gamma), U_{d-p}(\gamma)$ denote the flags 
\todo{Figure out what this exactly means}
corresponding to  $\rho(\gamma)$.
\end{definition}
\chapter{Upper bound}
\section{Proof of bound}
\begin{lemma}[Upper bound for dimension]\label{lem:upper_bound}
Let $\rho: \Gamma \to \SL(d, \mathbb R)$ be a projective Anosov representation. 
Then:
\[
    \dim_H(\xi^1 (\partial \Gamma) ) \leq \dim_F(\rho).
\]
\end{lemma}
\begin{remark}
    The idea of the proof of \cref{lem:upper_bound} is to find a covering whose Hausdorff content is dominated by the Dirichlet series of some functional $\Psi_s^p$, which will in turn imply that $\dim_H(\xi^1(\partial \Gamma)) \leq h_\rho(\Psi^p) $.
    Choosing then the most "effective" cover (i.e.\ the one which yields the smallest Hausdorff content up to a constant) we obtain that
    \[
        \dim_H(\xi^1(\partial \Gamma)) \leq h_\rho(\max_p \Psi^p)
    \]
    To obtain this we first cover $\xi^1(\partial \Gamma)$ by the bassins of attraction $\rho(\gamma) \cdot B_{\alpha_1, \alpha} (\rho(\gamma))$ for $\gamma \in \Gamma$ satisfying $|\gamma| = T$.
    Then we cover each bassin by an ellipsoid of axes lengths
    \[
        \frac{1}{\sin(\alpha)} \frac{\sigma_2}{\sigma_1}(\rho(\gamma)), \ldots, 
        \frac{1}{\sin(\alpha)} \frac{\sigma_d}{\sigma_1}(\rho(\gamma)).
    \]
    Finally we cover each ellipsoid by balls of some fixed radius $r>0$.
    It can be shown by comparing the series appearing in the Hausdorff content of each resulting cover that the most "effective" choice of $r$ depends only on the Hausdorff exponent $s > 0$ and in any case will be to have $r$ equal (up to a constant) to the the length of an axis of the ellipsoid, i.e.\
    \[
        r \in \left\{
            \frac{1}{\sin(\alpha)} \frac{\sigma_2}{\sigma_1}(\rho(\gamma)), \ldots, 
            \frac{1}{\sin(\alpha)} \frac{\sigma_d}{\sigma_1}(\rho(\gamma)).    
        \right\}
    \]
    In particular, when $s \in [p-2, p-1]$, the most effective choice is $r = \sigma_p(\rho(\gamma))/\sigma_1(\rho(\gamma))$, whose Hausdorff content is dominated by the Dirichlet series of $\Psi_s^p$.
\end{remark}

\begin{proof}[Proof of \cref{lem:upper_bound}]
Let $p \in \llbracket 2, d \rrbracket$.
Then using \cref{prop:basin_covering}, \cref{lem:boundary_covering}, and \cref{lem:ellipsoid_covering} we have that for $T>0$ large enough, $\xi^1(\partial \Gamma)$ is covered by the family
\[
    \mathcal U_T = \left\{ \rho(\gamma) B_{\alpha_1, \alpha}(\rho(\gamma)) : |\gamma| = T \right\},
\]
and that each basin $\rho(\gamma) B_{\alpha_1, \alpha}(\rho(\gamma))$ is in turn covered by
\[
    2^{p-2} \cdot \frac{\sigma_p(g)^{p-2}}{\sigma_2(g) \cdots \sigma_{p-1}(g)}
\]
many balls of radius
\[
    \sqrt{d-1} \frac{1}{\sin \alpha} \frac{\sigma_p(g)}{\sigma_1(g)}.
\]
By the definition of the Hausdorff measure, for $s \geq 0$:
\begin{align*}
    \mathcal H^s(\xi^1(\partial \Gamma)) &\leq
    \sum_{|\gamma| = T}
        2^{2p+1} \cdot 
        \frac{\sigma_2(\rho(\gamma))}{\sigma_1(\rho(\gamma))} \cdots 
            \frac{\sigma_{p-1}(\rho(\gamma))}{\sigma_1(\rho(\gamma))}
        \left(
            \frac{\sigma_p(\rho(\gamma))}{\sigma_1(\rho(\gamma))}
        \right)^{-(p-2)}
        \left(
            \sqrt{d-1} \frac{1}{\sin \alpha} \frac{\sigma_p(\rho(\gamma))}{\sigma_1(\rho(\gamma))}
        \right)^s =\\
        &=
        2^{2p+1} \cdot \left( \frac{\sqrt{d-1}}{\sin \alpha}\right)^s  
        \sum_{|\gamma| = T} 
        \frac{\sigma_2(\rho(\gamma))}{\sigma_1(\rho(\gamma))} \cdots 
            \frac{\sigma_{p-1}(\rho(\gamma))}{\sigma_1(\rho(\gamma))}
        \left(
            \frac{\sigma_p(\rho(\gamma))}{\sigma_1(\rho(\gamma))}
        \right)^{s-(p-2)} =\\
        &=
        2^{2p+1} \cdot \left( \frac{\sqrt{d-1}}{\sin \alpha}\right)^s  
        \sum_{|\gamma| = T}
        e^{-\left( \alpha_{1 2} + \ldots + \alpha_{1 (p-1)} + (s - (p-2))\alpha_{1 p} \right)\rho(\gamma)}\\
        &=
        2^{2p+1} \cdot \left( \frac{\sqrt{d-1}}{\sin \alpha}\right)^s  
        \sum_{|\gamma| = T}
        e^{-\Psi_s^p(\rho(\gamma))}
\end{align*}
and thus
\begin{align*}
    \mathcal H^s(\xi^1(\partial \Gamma)) \leq
    2^{2p+1} \cdot \left( \frac{\sqrt{d-1}}{\sin \alpha}\right)^s
    \sum_{|\gamma| = T}
    e^{-\max_p \Psi_s^p(\rho(\gamma)) }\lesssim \sum_{|\gamma| = T} e^{-F_s(\rho(\gamma))}.
\end{align*}
To see that the above implies the upper bound, consider some $s > \dim_F(\rho)$.
By the definition of the Falconer dimension, this implies that the Dirichlet series corresponding to the Falconer functional converges:
\[
    \sum_{\gamma \in \Gamma} e^{-F_s(\rho(\gamma))} < \infty
\]
and in particular
\[
    \mathcal H^s(\xi^1(\partial \Gamma)) \leq 
    \lim_{T \to \infty} e^{-F_s(\rho(\gamma))} = 0.
\]
\end{proof}

\section{Lemmata}
\begin{definition}
    Let $V$ be a finite-dimensional $\mathbb R$-vector space.
    We consider a decomposition
    \[
        V = \mathbb R u_1 \bigoplus \cdots \bigoplus \mathbb R u_d
    \] 
    be a direct decomposition that is orthogonal with respect to a fixed inner-product over $V$.
    Given $\beta_2 \geq \ldots \beta_d > 0$, we define an ellipsoid with axes $u_1 \oplus u_p(g)$ and lengths $\beta_p$ to be the image of
    \[
        \left\{
            v = \sum_1^d v_i u_i\in V : \sum_2^d \left( \frac{v_j}{\beta_j} \right)^2 \leq 1
        \right\}
    \]
    through the projection $ V \to \mathbb P (V)$.
\end{definition}
\begin{figure}[h]
    \centering
    \includegraphics[width=0.4\textwidth]{ellipsoid.jpg}
    \caption{Depiction in $\mathbb R^3$ of an ellipsoid of $\mathbb P(\mathbb R^2)$}
    \label{fig:ellipsoid}
\end{figure}    

The following aims to be something along the lines of \cite*[Lemma 2.4]{pozzetti_anosov_2023}:
\begin{lemma}\label{lem:angle}
    Let $\rho: \Gamma \to \SL(d, \mathbb R)$ be a projective Anosov representation.
    For $\alpha > 0$ small enough, there exists $L>0$ such that for any geodesic ray $(a_j)_j$ through $e$ we have:
    \[
        \angle(U_1(\rho(a_i)), U_{d-1}(\rho(a_0))) > \alpha
    \]
    when $|a_i|, |a_0| > T$.
\end{lemma}
\begin{proof}
Assume the contrary for the shake of contradiction.
Then (see Figure \ref{fig:angle} ) for each $n>0$ there exists a geodesic ray  $a^n$ through $e$ such that 
\[
    |a_n^n|, |a_0^n| > n \text{ and }
    \angle(U_1(\rho(a_n^n)), U_1(\rho(a_0^n))) < \frac{1}{n}.
\]
Due to compactness of $\Gamma \cup \partial \Gamma$ we can find some subsequence  ${k_n}$ and $x,y \in \partial \Gamma$ such that $a^{k_n}_{k_n} \to x$, $a^{-k_n}_0 \to y$  and $x\neq y$.
Since the limit map is dynamics preserving, we have that
\[
    \angle (\xi^1(x), \xi^{d-1}(y)) = 0,
\]
which contradicts its transversality property.
\end{proof}
\begin{figure}[h]
    \centering
    \includegraphics[width=0.3\textwidth]{angle.jpg}
    \caption{Situation in \cref{lem:angle}}
    \label{fig:angle}
\end{figure}    

The following is \cite[Proposition 3.5]{pozzetti_anosov_2023}.
\begin{lemma}\label{lem:boundary_covering}
Let $\rho: \Gamma \to SL(d, \mathbb R)$ be projective Anosov.
Then for $\alpha > 0$ small enough, there exists some $T_0 > 0$ such that for all $T \geq T_0$ the family
\[
    \mathcal U_T = \left\{ \rho(\gamma) B_{\alpha_1, \alpha}(\rho(\gamma)) : |\gamma| = T \right\}
\]
is an open covering of $\xi^1(\partial \Gamma)$.
\end{lemma}
\begin{proof}
    Let $\alpha, T > 0$ be as in the statement of \cref{lem:angle} and $x \in \partial \Gamma$ be represented by a geodesic ray $(\gamma_j)_{j\geq 0}$ starting from $e$.
    Then $(\gamma_T^{-1} \gamma_j)_j$ is a geodesic ray starting from $(\gamma_T)^{-1}$ that passes through $e$, so
    \[
        \angle (U_1(\rho(\gamma_T^{-1}\gamma_j)), U_{d-1}(\rho(\gamma_T^{-1}))) > \alpha
    \]
    as implied by \cref{lem:angle}.
    Taking the limit $j \to \infty$ and using the equivariance of the limit map, we obtain
    \[
        \angle (\rho(\gamma_T^{-1})\xi^1(x), U_{d-1}(\rho(\gamma_T^{-1}))) > \alpha
    \]
    and thus $\xi^1(x) \in \rho(\gamma_T) \cdot B_{\alpha_1, \alpha}(\rho(\gamma_T))$.
\end{proof}

The following is \cite[Proposition 3.8]{pozzetti_anosov_2023}.
\begin{proposition}\label{prop:basin_covering}
For each $g\in \SL(d, \mathbb R), \alpha > 0$, 
the basin of attraction $g \cdot B_{\alpha_1, \alpha}(g)$ lies in the ellipsoid with axes $u_1(g) \oplus u_p(g)$ with lengths
\[
    \frac{1}{\sin \alpha} \cdot \frac{\sigma_p(g)}{\sigma_1(g)}
\]
\end{proposition}
\begin{proof}
Using the definition of the basin of attraction (see \cref{fig:projection} ), we have that $w = w_1 u_1(g^{-1}) + \cdots + w_d u_d(g^{-1}) \in B_{\alpha_1, \alpha}(g)$ if and only if
\[
    w_d^2 \geq (\sin \alpha)^2 \sum_1^d w_i^2.
\]
Considering now some $v = v_1 u_1(g) + \cdots + v_d u_d(g) \in g \cdot B_{\alpha_1, \alpha}(g)$
we have that
\begin{align*}
    w = g^{-1} v &= v_1 \sigma_1(g)^{-1} l_g^{-1} e_1(g) + \cdots v_d \sigma_d(g)^{-1} l_g^{-1} e_d(g)\\
    &= v_1 \sigma_1(g)^{-1} u_{d}(g^{-1}) + \cdots v_d \sigma_d(g)^{-1} u_1(g^{-1})
\end{align*}
where we used that $u_p(g^{-1}) = l_g^{-1} e_{d+1-p}$.
Hence
\[
    \sigma_1(g)^{-2} \cdot v_1^2 \geq (\sin a)^2 \sum_1^d \sigma_i(g)^{-2} v_i^2.
\]
\end{proof}
\begin{figure}[h]
    \centering
    \includegraphics[width=0.4\textwidth]{projection.jpg}
    \caption{Aid for \cref{prop:basin_covering}}
    \label{fig:projection}
\end{figure}

The following is \cite[Lemma 3.7]{pozzetti_anosov_2023}:
\begin{lemma}\label{lem:ellipsoid_covering}
For any $p \in \llbracket 2, d \rrbracket$, an ellipsoid in $\mathbb P(\mathbb R^d)$ of axes lengths $\beta_2, \cdots, \beta_d$ is covered by
\[
    2^{p - 2} \frac{\beta_2 \cdots \beta_{p-1}}{\beta_p^{p-2}}
\]
many (projected) balls of radius $\sqrt{d-1} \beta_p$.
\end{lemma}
\begin{proof}
We assume that $E$ is an ellipsoid about $\mathbb R e_1$, so it suffice to cover its intersection $E_1 = E \cap U_1 \subseteq \mathbb R^{d-1}$ with the affine chart $U_1 = \{ [x_1: \ldots, x_d ] \in \mathbb P(\mathbb R^d) : x_1 \neq 0 \}$.
Clearly $E_1 \subseteq [-\beta_2, \beta_2] \times \ldots \times [-\beta_d, \beta_d]$, so we proceed by covering the rectangle with side-lengths $2\beta_2, \ldots, 2\beta_d$.
Clearly each interval $(-\beta_j, \beta_j)$ is contained in the union of $\lceil \beta_j/\beta_p \rceil$ intervals of length $2\beta_p$, thus $E_1$ is contained in the union of
\[
    \left\lceil \frac{\beta_2}{\beta_p} \right\rceil \cdots \left\lceil \frac{\beta_{p-1}}{\beta_p} \right\rceil =
    \left\lceil \frac{\beta_2}{\beta_p} \right\rceil \cdots \left\lceil \frac{\beta_d}{\beta_p} \right\rceil
\]
many squares of side-length $2\beta_p$.
Since each such product is contained in a $(d-1)$-ball of radius $\sqrt{d-1} \beta_p$ we may use at most
\[
    \left\lceil \frac{\beta_2}{\beta_p} \right\rceil \cdots \left\lceil \frac{\beta_{p-1}}{\beta_p} \right\rceil \leq
    \sum_{i \in \{ 0,1 \}^{p-2} } \prod_{j = 2}^{p-1} \left( \frac{\beta_j}{\beta_p} \right)^{i_j} \leq 2^{p-2} \frac{\beta_2}{\beta_p} \cdots \frac{\beta_{p-1}}{\beta_p}
\]
many $(d-1)$-balls of radius $\sqrt{d-1}\beta_p$ to cover $E_1$.
\end{proof}

The following can be found in \cite[Proposition 3.3]{pozzetti_anosov_2023}:
\begin{proposition}
    Let $\rho: \Gamma \to SL(d, \mathbb R)$ be projective Anosov and $\alpha > 0$
    Then there exist $c_0, c_1 > 0$ that depends only on $\alpha$ and $\rho$ such that for all $\gamma \in \Gamma$:
    \[
        (\xi^1)^{-1}(B_{\alpha_1, \alpha}(\rho(\gamma))) \subseteq C_{c_0,c_1}^\infty(\gamma)
    \]
\end{proposition}
\begin{proof}
    We begin by noting that it suffices to show this for all but finitely many $\gamma \in \Gamma$, since then we may alter the constants to satisfy the wanted inclusion for the finitely many remaining $\gamma \in \Gamma$ as well. 
    Hence, we may assume that $|\gamma| \geq l_0$ where $l_0 > 0$ is such that 
    \[
        Ce^{-\mu l_0} < 1 \text{ and } {\sf a}_1(\gamma) \geq C|\gamma| - L.
    \]

    Suppose $x \in \partial \Gamma$ such that $\xi^1(x) \in B_{\alpha_1, \alpha}(\rho(\gamma))$, and consider a geodesic ray $a_j \to x$ starting from $a_0 = e$.
    To prove the result, it suffices to find constants $c_0, c_1$ independent of $\gamma$ and for which there exists a ($c_0$, $c_1$)-quasi-geodesic from $\gamma^{-1}$ to $x$ that passes through $e$ and stays at a bounded distance from $(a_j)_{j=0}^\infty$.

    Using the exponential convergence rate of $\xi^1(a_j) \to \xi^1(x)$ and the definition of $B_{\alpha_1, \alpha}(\rho(\gamma))$ we have that:
    \begin{align*}
        d(\xi^1(a_j), \xi^1(\gamma)) &\geq
        d(\xi^1(x), U_1(\rho(\gamma^{-1}))-d(\xi^1(a_j), \xi^1(x)))
        \geq \\
        &\geq
        d(\xi^1(x), U_{d-1}(\rho(\gamma^{-1}))-d(\xi^1(a_j), \xi^1(x)))
        \geq
        \sin \alpha - Ce^{-\mu j}
    \end{align*}
    which along with the uniform continuity of $\xi^1: \Gamma \cup \partial \Gamma \to \mathbb P(\mathbb R^d)$ this implies there exists some $\alpha' > 0$ and $L>0$ such that for all $j\geq L$:
    \[
        d(a_j, \gamma^{-1}) \geq \alpha'.
    \]
    Upon considering a large $L$, we may also assume that $|a_L| = L > l_0$. Note that both $\alpha'$ and $L$ do not depend on each $\gamma$ but only on $\rho$ and $\alpha$.

    Using a coarse geometric arguement, we can show that for all $j \geq L$
    \[
        d(\gamma^{-1}, a_j) > \alpha' \Rightarrow
        d([\gamma^{-1}, a_j], e) < \alpha''
    \]
    for some $\alpha''$ that depends only on $\Gamma$ and $\alpha'$, where $[a_j, \gamma^{-1}]$ denotes any geodesic segment connecting $\gamma^{-1}$ and $a_j$.
    Indeed, \cite[Lemme 2.17]{ghys2013groupes} states that 
    $d([\gamma^{-1}, a_j]) \leq (\gamma_j^{-1}, a_j)_e + \delta$
    where $\delta$ is the hyperbolicity constant of $\Gamma$.
    Thus
    \[
        d([\gamma^{-1}, a_j]) \leq \delta + \frac{d(a_j, e) + d(\gamma^{-1},e) + d(a_j, \gamma^{-1})}{2} \leq
        \delta + \frac{L + d(\gamma^{-1}, e) + \alpha'}{2}.
    \]

    Consider the concatenation $(a_j')_{j=L-K}^\infty$ of $[\gamma^{-1},a_L]$ and $[a_L, x]$.
    To find quasi-geodesic-constants that are uniform in $\gamma$, we note that for any $c_0 \geq 1, c_1 \geq 0$:
    \[
        c_0^{-1} |i - j| - c_1 \leq d(a_i', a_j') = d(a_i, a_j) \leq c_0 |i - j| + c_1 
        \text{ when } i,j \geq L \text{ or } i,j \leq L
    \]
    and that the upper bound follows trivially by the triangle inequality. 
    
    For the lower bound we proceed in two steps. 
    First we bound the distance of $\gamma^{-1} = a_{L-K}'$ to $a_{L+j}$ for $j\geq 0$:
    \begin{align*}
        d(a_{L-K}', a_{L+j}') 
        &\geq \nu (|a_{L+j}| - |\gamma^{-1}|) - c_0' -c_1'|\log(d(U_1(\rho(a_{L+j})), U_1(\rho(\gamma^{-1}))))| \geq\\
        &\geq \nu((L+j) + (K-L)) - c_0' -c_1'|\log(\sin a)| \geq\\
        &= c_0^{-1} (j+K) - c_1
    \end{align*}
    for $c_0 = \nu^{-1}, c_1 = c_0' + c_1'|\log(\sin \alpha)|$.
    The first inequality comes from \cite[Lemma 3.9]{pozzetti_anosov_2023}. For the second inequality we estimate $|\gamma^{-1}|$ from below using the triangle inequality.
    We are now ready to show that the concatenation $(a_j')_j$ is indeed a ($c_0$, $c_1$)-geodesic:
    \begin{align*}
        d(a_{L+j}, a_{L-i}) &\geq d(a_{L+j},a_{L_K}) - d(a_{L_K}, a_{L-i}) \geq
        c_0^{-1} (j+K) - c_1 - (K - i) \geq\\
        &\geq c_0^{-1} (j+i) - c_1.
    \end{align*}

    Note however that $(a_j')$ does not necessarily lie in $C_\infty^{c_0, c_1}$ since it may not pass through $e$.
    For this reason we some $L - K \leq i_0\leq L$ such that $|a_{i_0}| < \alpha''$, the existence of which is guaranteed by the fact that $d([\gamma^{-1}, a_L], \epsilon) < \alpha''$.
    We then consider alter $(a_j')$ at $i_0$ so that it passes through $e$ to obtain 
    \begin{align*}
        a_j''=
        \begin{cases}
            a_j & \text{for } j\neq i_0 \\
            e & \text{for } j = i_0
        \end{cases}      
    \end{align*}
    which is a ($c_0, c_1 + \alpha''$)-quasigeodesic passing from $e$ and converging to $x$.
\end{proof}
\begin{figure}[h]
    \centering
    \includegraphics[width=0.8\textwidth]{cone.jpg}
\end{figure}
\chapter{Lower bound}
\section{Busemann cocyle and Patterson-Sullivan measures}
We denote with $\Pi$ the set of simple positive roots, and for $\Theta \subseteq \Pi$ we consider the Levi-Anosov subspace of $\mathfrak a$
\[
    \mathfrak a_\Theta = \bigcap_{\alpha \notin \Theta} \ker \alpha,
\]
which in particular admits $\{ \omega_{\alpha_i} : \alpha_i \in \Theta \}$ as a basis.
\begin{definition}
    Let $\Theta \subseteq \Pi$. We define the Busemann cocycle
    \[
    b : \PSL(d, \mathbb R) \times \mathcal F_\Theta \to \mathfrak a_\Theta
    \]
    as the unique element $b(g, k P_\Theta) \in \mathfrak a_\Theta$ such that
    \[
    g k \in K e^{b(g, k P_\Theta)} N.
    \]
    where $N = \{ n \in \SL(d, \mathbb R) : n_{ij} = 0 \text{ for } i > j, n_{ii} = 1 \text{ for all } i \}$ is the unipotent group of upper subgroup of upper triangular matrices with $1$s on the diagonal, and $P_\Theta$ is the parabolic subgroup of $\PSL(d, \mathbb R)$ corresponding to $\Theta$, i.e.\ $\mathcal F_\Theta = \PSL(d, \mathbb R)/P_\Theta$.
\end{definition}
\begin{lemma}
For $y \in \mathcal F_\Theta(\mathbb R^d)$ we have    
\[
    \omega_{\alpha_i}(b_\Theta(g, x)) =
    \log \frac{\| g v_1 \wedge \cdots g v_i \|}{ \| v_1 \wedge \cdots v_i \| }
    \text{ for all } \alpha_i \in \Theta
\]
for any basis $v_1, \ldots, v_i$ of $x^i \in \mathcal G_i(\mathbb R^d)$, where $\| \cdot \|$ denotes the norm on $\bigwedge^i \mathbb R^d$ induced by the euclidean inner product on $\mathbb R^d$, 
i.e.\ $ \langle v_1 \wedge \cdots \wedge v_k, w_1 \wedge \cdots \wedge w_k \rangle := \det(\langle v_i, w_j \rangle)$.
\end{lemma}

\begin{definition}
    We define
    \[
    \Lambda^k : \SL{d, \mathbb R} \to \SL{\Lambda^k \mathbb R^d}, \quad
    \Lambda^k : \mathcal G_k(\mathbb R^d) \to \mathbb P(\Lambda^k \mathbb R^d)
    \]
    as
    \[
    \Lambda^k(g)(v_1 \wedge \cdots \wedge v_k) = g v_1 \wedge \cdots \wedge g v_k
    \Lambda^k(\mathbb R v_1 \oplus \cdots \mathbb R v_k) =  v_1 \wedge \cdots \wedge v_k
    \]
\end{definition}
\begin{definition}
    For a discrete subgroup $\Gamma < \PSL(d, \mathbb R), \phi \in (\alpha_\Theta)^*$,
    a $(\Gamma, \phi)$-Patterson Sullivan measure on $\mathcal F_\Theta$ is a finite Radon measure $\mu$ such that for every $\gamma \in \Gamma$
    \[
        \frac{\d\gamma_* \mu}{\d\mu}(x) = e^{-\phi(b_\Theta(g^{-1},x))}, \text{ for all } x \in \mathcal F_\Theta(\mathbb R^d).
    \]
\end{definition}

\begin{lemma}\label{lem:busemann}
Let $\alpha > 0, \Theta \subseteq \Pi$.
There exists $K = K(\alpha) > 0$ such that for each $g \in \SL(d, \mathbb R), \sf{a_i} \in \Theta, y \in B_{\Theta, \alpha}(g), \phi \in \mathfrak a_\Theta$
\[
    |\phi (a(g) - b(g, y))| \leq K.
\]
\end{lemma}\begin{proof}
We begin by noting that it suffices to consider the case where $\phi = {\sf a}_k$
for $\sf a_k \in \Theta$, since $\{ \omega_i \}_{\sf a_i \in \Theta} $ 
is a basis for $\mathfrak a_\Theta^*$.

We first consider the case where $k=1$.
We recall that the first component of the Cartan projection coincides with the spectral norm of $g$, i.e.
\[
a_1(g) = \log \sup_{v \neq 0} \frac{\| g v\|}{\|v\|} = \log \|g k_2^{-1} e_1 \|
\] 
where $g = k_1 e^{a(g)} k_2$ is the Cartan decomposition of $g$.
Let $v = v_1 k_2^{-1} e_1 + \cdots + v_d k_2^{-1} e_d \in \mathbb R^d$ be such that $\|v\| = 1$ and $y = \mathbb R v$, we have

\begin{align*}
    |\omega_1(a(g) - b(g, y))| &=
    |\log \|g k_2^{-1} e_1 \| - \log \|g v \| | =\\
    &= |\log |e^{a_1(g)}| - \log \| e^{a_1(g)} v_1 k_1 e_1 + \cdots + e^{a_d(g)} v_d k_1 e_d \| | =\\
    &= \left| \log \left\| v_1 k_1 e_1 + e^{-{\sf a}_{12}(g)} v_2 k_1 e_2 + \cdots
    +  e^{-{\sf a}_{1d}(g)} v_d k_1 e_d \right\| \right| \leq\\
    &\leq |\log |v_1| | = \left| \log \sin( \angle(v, U_{d-1}(g^{-1}))) \right| \leq \left| \log \sin \alpha \right|.
\end{align*}

For the case where $\Theta = \{ {\sf a}_k \}$, we consider $\alpha' > 0$ such that
\[
\Lambda^k (B_{{\sf a}_k, \alpha}(g)) \subseteq B_{{\sf a}_k, \alpha'}(\Lambda^k g)
\]
Then using the case $k=1$ we have that
\begin{align*}
    |\omega_k(a(g) - b(g, y))| &=
    |\omega_1(a(\Lambda^k g) - b(\Lambda^k g, \Lambda^k y))| \leq \left| \log \sin \alpha' \right| .
\end{align*}
\end{proof}

\section{Proof strategy}
Denoting with $d_\Gamma = \dim_H \xi_\rho^1(\partial \Gamma)$ the Hausdorff  dimension of the limit set, we shall outline how to obtain its lower bound by the Falconer dimension
\[
    d_\Gamma \geq h_\rho(F).
\]

First we recall that $F_s(a) = \max \{ \Psi_s^p(a) : p \in \llbracket 2, d \rrbracket\}$ and in particular $h_\rho(F) \leq h_\rho(\Psi^{d_\Gamma + 1})$.
Thus the lower bound will follow once we have shown that
$$d_\Gamma \geq h_\rho(\Psi^{d_\Gamma + 1}).$$

Noting that $(s+1) J_{d_\Gamma^u} \leq \Psi_{s+d_\Gamma}^{d_\Gamma + 1}$, the above bound will follow as soon as we have shown that
\begin{equation}\label{eq:JacobianBound}
    h_\rho(J_{d_\Gamma}) \leq 1.\tag{LB}
\end{equation}
To obtain inequality (\ref{eq:JacobianBound}), one uses the method of Patterson-Sullivan-Quint, which may be summed up in
\begin{align*}
    \text{There exists a } (\phi, \rho)\text{-Patterson-Sullivan measure on } \mathcal F_\Theta(\mathbb R^d) 
    \Rightarrow h_\rho(\phi) \leq 1,
\end{align*}
where $\phi \in \mathfrak a_\Theta$ and $ \Theta \subseteq \Pi$.
The property that we will need of our measure is that there exists a collection of open sets ${U_\gamma}_\gamma \in \Gamma$ such that
\begin{align*}\label{eq:GoodMeasure}
    \mu(U_\gamma) \sim e^{-J_{d_\Gamma}^u(a(\rho(\gamma)))} \text{ and } 
    M := \sup_{n \in \mathbb N}
    \max
    \left\{\sharp A : A \subseteq \Gamma_n,  \bigcap_{\gamma \in A}U_\gamma \neq \emptyset \right\} < \infty\tag{MP}
\end{align*}
where $\Gamma_n = \{ \gamma \in \Gamma : |\gamma| = n \} $.
For the proof of the existence of a $(J_{d_\Gamma}^u, \rho)$-Patterson-Sullivan measure that satisfies the property (\ref{eq:GoodMeasure}) we refer to \cref{sec:MeasureExistence}, noting that the Zariski-density assumption is necessary only for the equivalence appearing on the left hand side of \cref{eq:GoodMeasure}.
Assuming that for the time being, below we outline the Patterson-Sullivan-Quint method of obtaining \cref{eq:JacobianBound}.

Indeed, we first obtain the uniform in $n$ bound:
\begin{align*}
    \sum_{\gamma \in \Gamma_n}
    e^{-J_{d_\Gamma}^u(a(\rho(\gamma)))} \lesssim
    \sum_{\gamma \in \Gamma_n}
    \mu(U_\gamma) \leq \frac{1}{M} \mu(\mathcal F_\Theta(\mathbb R^d)) < \infty
\end{align*}
along with the bound implied by the Anosov property of $\rho$:
\begin{align*}
J_{d_\Gamma}(a(\rho(\gamma))) \geq \mathsf a_{12} (a(\rho(\gamma))) \geq C|\gamma| - b
\end{align*}
to conclude that
\begin{align*}
\sum_{\gamma \in \Gamma} e^{-(s+1)J^u_{d_\Gamma}(a(\rho(\gamma)))} &=
\sum_{n\geq 0} \sum_{\gamma \in \Gamma_n} e^{-s J^u_{d_\Gamma}(a(\rho(\gamma)))}
    e^{J^u_{d_\Gamma}(a(\rho(\gamma)))} \lesssim
\sum_{n\geq 0} e^{-s(Cn - b)} < \infty
\end{align*}
which holds for any $s>0$, and thus \cref{eq:JacobianBound} holds.

\section{Existence of Patterson-Sullivan measure}\label{sec:MeasureExistence}
\begin{definition}
    For $p \in \llbracket 2, d \rrbracket$, we denote the $p$-th unstable Jacobian $J_p^u \in \mathfrak a^*$ by
    \[
    J_p^u = (p+1) \omega_{{\sf a}_1} - \omega_{{\sf a}_{p+1}} = 
    {\sf a}_{12} + \cdots + {\sf a}_{1(p+1)}.
    \]

\end{definition}
\begin{definition}
    Let $V \in \mathcal G_{p+1}{\mathbb R^d}$ and $l \in \mathbb P(V)$.
    Using the canonical identification $T_l \mathbb P(V) \simeq \hom (l, V/l)$, we define the density $|\Omega_{l, V}|$ on $\bigwedge^p T_l \mathbb P(V)$ by
    \[
        |\Omega_{l,V}|(\phi_1, \ldots, \phi_p) = 
        \frac{\| v \wedge \tilde \phi_1(v) \wedge \cdots \wedge \tilde \phi_p (v) \|}{\|v\|^{p+1}}
    \]
    for any $v \in l - \{ 0\}$, where $\tilde \phi_1, \ldots \tilde \phi_p \in \hom(l, V)$ are such that $\phi_i = \tilde \phi_i + \hom(l, l)$ and $\| \cdot \|$ denotes the norm on $\bigwedge^{p+1} \mathbb R^d$ induced by the euclidean inner product.
\end{definition}

The following is \cite[Proposition 6.4]{pozzetti_anosov_2023}:
\begin{proposition}
Assume that $\xi^1_\rho(\partial \Gamma)$ is a Lipschitz submanifold of dimension $d_\Gamma$.
Then there exists a $(\rho(\Gamma), J_{d_\Gamma}^u)$-Patterson-Sullivan measure on $\mathcal F_{1,d_\Gamma + 1}$.
\end{proposition}
\begin{proof}
    By Rademacher's theorem, $\xi^1_\rho(\partial \Gamma)$ has a well-defined Lebesgue measure class, and Lebesgue-almost every $\xi^1_\rho(x) \in \xi^1_\rho(\partial \Gamma)$ admits a well-defined tangent space $T_{\xi^1_\rho(x)} \xi^1_\rho(\partial \Gamma)$.
    Considering such a $\xi^1_\rho(x)$ we let
    \[
        \pi : \hom(\xi^1_\rho(x), \mathbb R^d) \to \hom(\xi^1_\rho(x), \mathbb R^d/\xi^1_\rho(x)) \simeq T_{\xi^1_\rho(x)} \xi^1_\rho(\partial \Gamma),
    \]
    and
    \[
        x^{d_\Gamma + 1} = \pi^{-1} (T_{\xi^1_\rho(x)} \xi^1_\rho(\partial \Gamma)) \xi^1_\rho(x) \in 
        \mathcal G_{d_\Gamma + 1} (\mathbb R^d),
    \]
    for which one can show that
    \[
        T_{\xi^1_\rho(x)} \xi^1_\rho(\partial \Gamma) \simeq 
        \hom(\xi^1_\rho(x), \mathbb R^d / \xi^1_\rho(x)) \simeq
        \hom(\xi^1_\rho(x), x^{d_\Gamma + 1} / \xi^1_\rho(x)).
    \]
    In this notation, we shall define (Lebesgue-almost eeverywhere) the map
    \begin{align*}
        \zeta_\rho: \xi_\rho^1(\partial \Gamma) \to \mathcal F_{1, d_\Gamma + 1} (R^d), \quad 
        \zeta_\rho(\xi_\rho^1(x)) = (\xi_\rho^1(x), x^{d_\Gamma + 1}).
    \end{align*}

    We now define the non-negative density on $\xi_\rho^1(\partial \Gamma)$
    \[
        \mu_{\xi_\rho^1(x)} = |\Omega_{\zeta_\rho(\xi_\rho^1(x))}|
    \]
    which satisfies
    \[
        \frac{\d \gamma_*\mu}{\d\mu}(\xi) = \frac{\d (\rho(\gamma)^{-1})^*\mu}{\d\mu}(\xi) =
        e^{-J^u_{d_\Gamma + 1}(b_\Theta(\rho(\gamma)^{-1}, \zeta(x))))},
    \]
    where the term on the left-hand side involves measures, the term between the two equalities involves densities, and $\Theta = \{1, d_\Gamma + 1\}$.
    Indeed, for $\phi_1, \ldots, \phi_{d_\Gamma} \in T_{\xi_\rho^1(x)}\xi_\rho^1(\partial \Gamma)$:
    \begin{align*}
        &(\rho(\gamma)^*\mu)_{\xi_\rho^1(x)} (\phi_1, \ldots, \phi_{d_\Gamma})\\
        &=
        \mu_{\rho(\gamma)\xi_\rho^1(x)} (\rho(\gamma)\phi_1\rho(\gamma)^{-1}, \ldots, \rho(\gamma)\phi_{d_\Gamma} \rho(\gamma)^{-1}) =\\
        &=
        \frac{\|\rho(\gamma) \xi_\rho^1(x) \wedge \rho(\gamma) \phi_1(\xi_\rho^1(x)) \wedge \cdots \wedge \rho(\gamma) \phi_{d_\Gamma}(\xi_\rho^1(x))\|}{\| \rho(\gamma) \xi_\rho^1(x) \|^{d_\Gamma + 1}} =\\
        &=
        \frac{\|\rho(\gamma) \xi_\rho^1(x) \wedge \rho(\gamma) \phi_1(\xi_\rho^1(x)) \wedge \cdots \wedge \rho(\gamma) \phi_{d_\Gamma}(\xi_\rho^1(x))\|}
        {\|\xi_\rho^1(x) \wedge \phi_1(\xi_\rho^1(x)) \wedge \cdots \wedge \phi_{d_\Gamma}(\xi_\rho^1(x))\|} \cdot
        \frac{\|\xi_\rho^1(x) \wedge \phi_1(\xi_\rho^1(x)) \wedge \cdots \wedge \phi_{d_\Gamma}(\xi_\rho^1(x))\|}
        {\|\rho(\gamma) \xi_\rho^1(x)\|^{d_\Gamma+1}}\cdot
        \frac{\| \rho(\gamma)\xi_\rho^1(x)\|^{d_\Gamma+1}}{\|\xi_\rho^1(x)\|^{d_\Gamma+1}}
         =\\
         &=
         e^{\omega_{d_\Gamma}(b_\Theta(\rho(\gamma), \zeta_\rho(\xi_\rho^1(x))))} \cdot
         \mu_{\xi_\rho^1(x)}(\phi_1, \ldots, \phi_{d_\Gamma}) \cdot
         e^{-(p+1)\omega_1(b_\Theta(\rho(\gamma), \zeta_\rho(\xi_\rho^1(x))))} =\\
         &=
         e^{-J_{d_\Gamma}^u(b_\Theta(\rho(\gamma)^{-1}, \zeta(\xi_\rho^1(x)))} \mu_{\xi_\rho^1(x)}(\phi_1, \ldots, \phi_{d_\Gamma}).
    \end{align*}
    
    Finally, we let $\nu = {\zeta_\rho}_* \mu$, which is the wanted Patterson-Sullivan measure on $\mathcal F_{1, d_\Gamma + 1} (\mathbb R^d)$, since for $f \in C_c(\mathcal F_{1, d_\Gamma + 1} (\mathbb R^d))$:
    \begin{align*}
        \int_{\mathcal F_{1, d_\Gamma + 1} (\mathbb R^d)} f \d(\gamma_* {\zeta_\rho}_* \mu) &=
        \int_{\xi_\rho^1(\partial \Gamma)} f \circ \gamma \circ \zeta_\rho \d \mu =
        \int_{\xi_\rho^1(\partial \Gamma)} f \circ \zeta_\rho \circ \gamma \d \mu =\\
        &=
        \int_{\xi_\rho^1(\partial \Gamma)} f \circ \zeta_\rho(\xi) e^{-J_{d_\Gamma}^u(b_\Theta(\rho(\gamma)^{-1}, \zeta(\xi_\rho^1(x)))} \d \mu(\xi_\rho^1(x)) =\\
        &=
        \int_{\mathcal F_{1, d_\Gamma + 1} (\mathbb R^d)} 
        f(y) e^{-J_{d_\Gamma}^u(b_\Theta(\rho(\gamma)^{-1}, y)} \d ({\zeta_\rho}_*\mu)(y)
    \end{align*}
\end{proof}

Before giving the next definition, we recall that the annihilator annihilator of an element $y \in \mathcal F_F{i \Theta}(\mathbb R^d)$ is the set of partial flags that are not transverse to $y$, that is:
\[
    \mathrm{Ann}(y) = \left\{ x \in \mathcal F_{\Theta}(\mathbb R^d) : x^\theta \cap y^{d- \theta} \neq 0 \text{ for some } \theta \in \Theta \right\}.
\]
\begin{definition}
Let $\rho: \Gamma \to \SL(d,\mathbb R)$ be a linear representation, $\Theta \subseteq \Pi$ and $\mu$ a measure over $\mathcal F_\Theta(\mathbb R^d)$.
We say that $\rho$ is $\mu$-irreducible there is no element in $\mathcal F_{i \Theta}(\mathbb R^d)$, whose annihilator is of full measure, i.e.\ for all $y \in \mathcal F_{i \Theta}(\mathbb R^d)$:
\[
    \mu(\mathrm{Ann}(y)) < \mu(\mathcal F_{\Theta}(\mathbb R^d)).
\]
\end{definition}

\begin{example}
    If $\rho(\Gamma)$ is Zariski-dense in $\SL (d, \mathbb R)$, then $\rho$ is $\mu$-irreducible for any $\rho$-quasi-equivariant measure $\mu$, and in particular for any $(\rho(\Gamma), \phi)$-Patterson-Sullivan measure.
\end{example}

\begin{remark}
    The reason that we introduce the concept of $\mu$-irreducibility is that for any $\mu$-irreducible representation $\rho:\Gamma \to \SL(d,\mathbb R)$, there exist $\alpha, \kappa > 0$ such that $\mu(B_{\Theta, \alpha}(\rho(\gamma))) \geq k$ for all $\gamma \in \Gamma$.

    Indeed, if this were not the case, then there would exists a sequence $\alpha_n \searrow 0$ and $\gamma_n \in \Gamma$ such that
    \[
        \mu(B_{\theta, \alpha}(\rho(\gamma))) \leq \frac{1}{n}.
    \]
    Due to the compactness of $\mathcal F_{\Theta} (\mathbb R^d)$, up to considering a subsequence, we may assume that the reppeling flags or $\rho(\gamma_n)$ converge to some $\xi \in \mathcal F_{\Theta} (\mathbb R^d)$:
    \[
        (U_{d-i}(\rho(\gamma_n)^{-1}))_{\mathsf a_i \in \Theta} \to \xi
    \]
    
    In that case, the complements $B_{\Theta, \alpha_n}^c(\rho(\gamma_n))$ will converge to the annihilator of $\xi$, in the sense:
    \[
        \limsup_n B_{\Theta, \alpha_n}^c(\rho(\gamma_n)) \subseteq \mathrm{Ann}(\xi).
    \]
    Indeed, let $y\in \limsup_n  B_{\Theta, \alpha_n}^c(\rho(\gamma_n))$ and consider a subsequence $k_n$ such that $y\in B_{\Theta, \alpha_n}^c(\rho(\gamma_{k_n}))$.
    By the very definition of $B_{\Theta, \alpha_n}(\rho(\gamma_n))$, there exists some $p$ such that up to considering a subsequence of $k_n$,
    \[
        \angle (y^p, U_{d-p}(\rho(\gamma_n)^{-1})) \leq \alpha_n
    \]
    holds.
    Taking the limit as $n \to \infty$, we have that $y^p \cap \xi^{d-p} \neq 0$ and hence $y \in \mathrm{Ann(\xi)}$.

    Using a measure-theoretic arguement we conclude that $\mathrm{Ann}(\xi)$ is of full measure, which contradicts the $\mu$-irreducibility of $\rho$:
    \[
        \mu(\mathrm{Ann}(\xi)) \geq
        \mu (\limsup_n B_{\Theta, \alpha_n}^c(\rho(\gamma_{k_n})) ) \geq 
        \limsup_n \mu (B_{\Theta, \alpha_n}^c(\rho(\gamma_{k_n}))) =
        \mu(\mathcal F_\Theta (\mathbb R^d)).
    \]
\end{remark}

\begin{lemma}
    Let $\rho:\Gamma \to \SL(d, \mathbb R)$ be a representation and $\mu^\phi$ be a ($\rho(\Gamma), \phi$)-Patterson-Sullivan measure.
    If $\rho(\Gamma)$ is $\mu$-irreducible, then there exists some $\alpha_0 > 0$, such that for any $\alpha \in (0, \alpha_0)$, there's some $k = k(\alpha) > 0$ for which
    \[
        \frac{1}{k} e^{-\phi(a(\rho(\gamma)))} 
        \leq 
        \mu^\phi(\rho(\gamma) B_{\Theta, \alpha}(\rho(\gamma))) 
        \leq
        k e^{-\phi(a(\rho(\gamma)))} 
    \]
    for all $\gamma \in \Gamma$. 
\end{lemma}
\begin{proof}
    Let $\alpha_0, k > 0$ be as in the remark preceeding the statement of the lemma.
    As noted in \cref{lem:busemann}, there exists some $K = K(\alpha_0, \phi) > 0$ such that for any $\alpha \in (0, \alpha_0)$ and $y \in B_{\Theta, \alpha}(\rho(\gamma))$:
    \[
        |\phi(a(\rho(\gamma)) - b(\rho(\gamma), y))| \leq K,
    \]
    from which we obtain the upper bound
    \begin{align*}
        \mu^\phi(\rho(\gamma) B_{\Theta, \alpha}(\rho(\gamma))) &=
        (\rho(\gamma^{-1})_*\mu^\phi)(B_{\Theta, \alpha}(\rho(\gamma))) =
        \int_{\mathcal F_\Theta(\mathbb R^d)} e^{-\phi(b(\rho(\gamma), y))} d\mu^\phi(y)\leq\\
        &\leq
        e^{-K} \mu^\phi(\mathcal F_\Theta (\mathbb R^d)) e^{-\phi(a(\rho(\gamma)))}.
    \end{align*}
    
    Similarly we obtain the lower bound
\end{proof}

\appendix
\chapter{Tangent space to the Grassmanian}
Let $V$ be a $d$-dimensional real vector space. 
We denote with $\mathcal G_{k}(V)$ the Grassmanian of $k$-dimensional subspaces of $V$.
Our first objective is to find a convenient way to express its tangent space.
\begin{proposition}
We have the following canonical identification:
\begin{align*}
    \hom(W, V/W) &\simeq T_W \mathcal G_{k}(V)\\
    \phi &\mapsto \frac{\d}{\d t} \Big|_{t=0} \Gamma(t \phi)
\end{align*}
where $\Gamma(\phi) = (Id+\phi)(W)$ is the graph of $\phi$.
\end{proposition}
\begin{proof}
    We will consider the map
    \begin{align*}
        F: \mathrm{Injhom}(W, V) \to \mathcal G_{k}(V), \quad \phi \mapsto \phi(W).
    \end{align*}
    whose derivative is given by:
    \begin{align*}
        d_I F (\phi) = F \left(\frac{\d}{\d t} \Big|_{t=0} (I + t\phi)\right)
        = \frac{\d}{\d t} \Big|_{t=0}  \left( I + t\phi (W) \right)
        = \frac{\d}{\d t} \Big|_{t=0}  \Gamma(t \phi).
    \end{align*}
    The result will follow as soon as we have shown that $d_I F$ is surjective and that $\ker d_I F = \hom(W, W)$.
    
    To show that it is surjective, we consider a $(d-k)$-dimensional subspace $W' \in \mathcal G_{d-k}(V) $ that is complementary to $W$, i.e.\ $V = W \oplus W'$.
    Denoting with $U_{W'} = \{ Z \in \mathcal G_k (V): Z \cap W' = 0 \}$, we recall the corresponding chart:
    \begin{align*}
        \Psi: \hom(W, W') &\to U_{W'}\\
        \phi &\mapsto \Gamma(\phi).
    \end{align*}
    Surjectivity of $d_I F$ now follows by the fact that
    \[
        d_I F(\phi) = \frac{\d}{\d t} \Big|_{t=0} \Gamma(t \phi) = d_0 \Psi(\phi).
    \]

    To show that $\ker d_I F = \hom(W, W)$, we first note that clearly $\ker \d_I F \supseteq \hom(W, W)$.
    Equality then follows by the fact that $\dim \hom(W, W) = \dim \ker \d_I F$, which is a direct consequence of the surjectivity.
\end{proof}

Note that another way to prove the above identification throught the fact that the Grassmanian is a homogeneous space of $\GL(d, \mathbb R)$, giving us the diffeomorphism
\begin{align*}
    \GL(V) / \mathrm{St}_{GL(V)} W &\to \mathcal G_k(V)\\
    [g] &\mapsto gW,
\end{align*}
where $\mathrm{St}_{GL(V)} W = \{ g \in \GL(V) : gW = W \}$ is the stabilizer of $W$.
Thus an expression for the tangent space at $W$ may be obtained by differentiating the map above at the identity coset:
\begin{align*}
    \hom(W, V/W) \simeq \hom(V, V) / \hom(W, W) &\simeq T_W \mathcal G_k(V).
\end{align*}
In particular the above gives us a direct proof that the kernel of the map constructed in the previous proof is indeed $\hom(W, W)$.

Our second objective is to identify subspaces of $T_l \mathbb P(V)$ with subspaces of $V$, by considering the first as projectivisation of the second.
More concretely, we shall consider the space
\[
    \mathcal P = \{ (l, P) : l \in \mathbb P(V), P \in \mathcal G_k( T_l \mathbb P(V) )  \}
\]
as a homogenous space of $\SL(V)$, where the action is given by
\begin{align*}
g \cdot (l, P) = (g l, \d_l g (P) = g \pi^{-1}(P) g^{-1} + \hom(gl, gl)).
\end{align*}
where we use the identification of $T_l \mathbb P(V)$ with $\hom(l, V/l)$ as above and denote with $\pi : \hom(l, V) \to \hom(l, V/l)$ the canonical projection.
For the sake of completeness, we outline the calculation of the differential:
\begin{align*}
    \hom(l, V/l) &\to T_l \mathbb P(V) &\to &T_{gl} \mathbb P(V) &\to \hom(gl, V/gl)\\
    \phi &\mapsto \frac{\d}{\d t} \Big|_{t=0} g(I + t \tilde \phi)(l) &\mapsto &\frac{\d}{\d t} \Big|_{t=0} (I + t g\tilde \phi g^{-1})(g l) &\mapsto g \tilde \phi g^{-1} + \hom(gl, gl)
\end{align*}
where $\phi \in \hom(l, V/l), \tilde \phi \in \hom(l, V)$ such that $\tilde \phi + \hom(l, l) = \phi$.

We are now ready to express the needed identification:
\begin{proposition}
We have the following $\SL(V)$ equivariant identification:
\begin{align*}
    \mathcal P &\to \mathcal F_{1, k+1}(V)\\
    (l, P) &\mapsto (l, \pi^{-1}(P)l)\\
    (l, \hom(l, Q/l)) &\mapsfrom (l, Q).
\end{align*}
where $\pi : \hom(l, V) \to \hom(l, V/l)$ is the canonical projection.
\end{proposition}
\begin{proof}
    We begin by showing that the left-to-right direction of the map is well-defined.
    For this, we first need to check that for $(l,P) \in \mathcal P$, we have that $\dim \pi^{-1}(P)l = k + 1$.
    Indeed, we have that $\dim \pi^{-1}(P) = k+1$ as implied by the rank-nullity theorem for $\pi: \pi^{-1}(P) \to P$.
    The result then follows by the fact that $\pi^{-1}(P)l = T_1(l) \oplus \cdots \oplus T_{k+1}(l)$ for any base $T_1, \ldots, T_{k+1}$ of $\pi^{-1}(P)$.
    The second thing to check is that $ l \leq \pi^{-1}(P)l$, which holds since $\ker \pi = \hom(l,l) \leq \pi^{-1}(P)$.

    To see that the two directions above are inverse to each other, we begin by examining the right-to-left-to-right composition:
    \[
    (l, Q) \mapsto (l, \pi(\hom(l, Q))) \mapsto (l, \pi^{-1}\pi(\hom(l, Q))) = (l, \hom(l, Q)l) = (l, Q).
    \]
    and for the left-to-right-to-left composition
    \[
        (l, P) \mapsto (l, \pi^{-1}(P)l) \mapsto (l, \hom(l, \pi^{-1}(P)l)),
    \]
    so it suffices to show that $\hom(l, \pi^{-1}(P)l/l) = P$.
    Indeed, for $\pi^{-1}(P) = \mathbb R T_1 \oplus \cdots \oplus \mathbb R T_k$, we have that
    \begin{align*}
        \hom(l, \pi^{-1}(P)l/l) &=
        \hom(l, \pi^{-1}(P)l)/\hom(l,l) =
        \left( \oplus_i \hom(l, T_i(l))/\hom(l,l)\right) =\\
        &= \left(\oplus_i \mathbb R T_i\right)/\hom(l,l) 
        = \pi^{-1}(P)/\hom(l,l) = P.
    \end{align*}

    For the equivariance, the calculations has as follows:

    \begin{tikzcd}
        (l, P) \arrow[r, mapsto] \arrow[d, mapsto, "g"] & (l, \pi^{-1}(P)l) \arrow[d, mapsto, "g"] \\
        (gl, g \pi^{-1}(P)g^{-1} + \hom(gl,gl)) \arrow[r, mapsto] & (gl, (g \pi^{-1}(P)g^{-1})(gl))= (gl, g\pi^{-1}(P)l)
    \end{tikzcd}
\end{proof}

\chapter{Irreducible actions problem}
The matter of this chapter has to do with an obstruction, found in the proof of this lemma: 
\begin{lemma}[Lemma 6.8 in \cite{pozzetti_anosov_2023}]
Let $\Gamma$ be a hyperbolic group and $\eta:\Gamma \to \PGL(d, \mathbb R)$ be a strongly irreducible projective Anosov representation such that $\xi_\eta(\partial \Gamma)$ is homeomorphic to $S^{d_\Gamma}$, and which admits a measurable $\eta$-equivariant section
$\zeta: \partial \Gamma \to \mathcal F_{\{ \sf a_1, \sf a_{d_\Gamma + 1}\}} (\mathbb R^d)$.
Then $\eta$ is $\mu$-irreducible for any $(\eta(\Gamma), \phi)$-Patterson-Sullivan measure $\mu$ on $\mathcal F_{\{ \sf a_1, \sf a_{d_\Gamma + 1}\}} (\mathbb R^d)$.
\end{lemma}

For convenience, we recall that a linear representation $\rho: \Gamma \to \GL(d, \mathbb R)$ is strongly irreducible if there is no proper $\rho(\Gamma)$-invariant subspace of $\mathbb R^d$,
and it is $\mu$-irreducible if there is no element in $\mathcal F_{i \Theta}(\mathbb R^d)$, whose annihilator is of full measure.

In what follows, we will show that the above lemma is false, by providing a counterexample.
Let $\Gamma$ be a uniform lattice of $\SU(2,1)$ (i.e.\ acts convex cocompactly ), and $\eta: \Gamma \to \SL(\mathfrak{su}(2,1))$ be the restriction of the adjoint representation, i.e. $\eta(\gamma) = \Ad_{\gamma}$ for all $\gamma \in \Gamma$.

For convenience, we recall the definition of a uniform lattice:
\begin{definition}
    Let $G$ be a locally compact group.
    A uniform lattice is a discrete subgroup $\Gamma \leq G$ that is co-compact, i.e.\ $G/\Gamma$ is compact.  
\end{definition}
\begin{remark}
    When $G = \Isom(X)$ is the isometry group of a complete Riemannian manifold $X$, and $\Gamma$ is a uniform lattice of $G$,then it acts properly discontinuously and cocompactly on $X$.
\end{remark}

We begin by showing proving the Anosov property of $\eta$.
\begin{proposition}\label{prop:counterexample_anosov}
    $\eta$ is projective Anosov.
\end{proposition}
\begin{proof}
Let $\gamma \in \Gamma$.
Since $\gamma \in \SU(2,1)$, we have that
\[
\gamma = k_1 \exp\left( r(\gamma) x_0 \right) k_2
\]
for $x_0$ a fixed non-zero in the Weyl-chamber $\mathbb R x_0$ of $\su(2,1), r(\gamma) \in \mathbb R$ and $k_1, k_2 \in \U(2)$.
Then by the definition of a uniform lattice, we have that $\Gamma$ acts properly discontinuously and cocompactly, which means that the inclusion $\Gamma \hookrightarrow \SU(2,1)$ is projective Anosov (since $\SU(2,1)$ is of rank 1).
Thus there exist constants $L \geq 1, b \geq 0$ such that for all $\gamma \in \Gamma$:
\[
r(\gamma) \geq {\sf a}_1(x_0)^{-1}(L|\gamma| - b) = L'|\gamma| - b'.
\]
Note that ${\sf a}_1(x_0) > 0 $ since $x_0$ is in the interior of the Weyl-chamber $\mathbb R x_0$.

Letting $k_1' = Ad_{k_1}, k_2' = Ad_{k_2}$ and $K' \leq \SL(\su(2,1))$ be a maximal comapct subgroup containing them, we have that:
\[
\eta(\gamma) = \Ad_{\gamma} = k_1' \Ad_{\exp(r(\gamma) x_0)} k_2' = k_1' \exp(r(\gamma) \ad_{x_0}) k_2'.
\]

Thus
\begin{align*}
    {\sf a}_1(\mu(\eta(\gamma))) = r(\gamma) {\sf a}_1(\ad_{x_0}) \geq (L'|\gamma| - b'){\sf a}_1(x_0)
\end{align*}
which is Anosov because ${\sf a}_1(\ad_{x_0}) > 0$, as can be seen by concrete calculations.
\end{proof}

Before giving an expression for the projective part of the limit map of $\eta$, we make a few observations regarding Gromov boundary of $\Gamma$.
In particular, we claim that since $\Gamma$ is a uniform lattice os $\SU(2,1)$, we have that $\partial \Gamma$ is homeomorphic to $\SU(2,1)/P$, where $P$ is a parabolic subgroup of $\SU(2,1)$, and it coincides with the stabilizer of some isotropic line $l \in \partial_\infty \mathbb H^2_\mathbb C$.

Indeed, for a uniform lattice $\Gamma$ of the isometry group $G$ of a homogenous $G$-space $X$, the Milnor-Švarc lemma implies that for any $x_0 \in X$, the map $\Gamma \to X, \gamma \mapsto \gamma x_0$ is a quasi-isometry.
In our case $G = \SU(2,1)$ and $X = \mathbb H^2_\mathbb C$ is a hyperbolic metric space, so it the quasi-isometry extends to a homeomorphism $\partial \Gamma \to \partial H^2_\mathbb C$ of the Gromov-boundaries.
On the other hand, the action of $\SU(2,1)$ on $\partial_\infty \mathbb H^2_\mathbb C$ is transitive, so we have that $\partial \mathbb H^2_\mathbb C \simeq \SU(2,1)/P$ where $P$ is the stabilizer of a point in $\partial \mathbb H^2_\mathbb C$.
In fact, we have that $P$ is a parabolic subgroup of $SU(2,1)$.
The combination of the above, along with the fact that the geometric and the Gromov boundaries agree in the case of $\mathbb H^2_\mathbb C$, we deduce that $\partial \Gamma \simeq \SU(2,1)/P$.\todo{Add proofs of these.}

To calculate the projective part of the limit map, we shall show that there exists a unique $\SU(2,1)$-equivariant map from $\partial \Gamma$ to $\mathbb P (\mathbb R^d)$.
The uniqueness follows from the following characterisation of limit maps: 
\begin{lemma}\label{lem:equivariance_uniqueness}
    Let $\rho: \Gamma \to \SL(d, \mathbb R) $ be a strongly irreducible projective Anosov representation, and denote with $\xi_\rho: \partial \Gamma \to \mathbb P(\mathbb R^d)$ its limit map.
    Then $\xi^1_\rho$ is the unique continuous, $\rho(\Gamma)$-equivariant map from $\partial \Gamma$ to $\mathbb P(\mathbb R^d)$.
\end{lemma}
\begin{proof}
    Let $\eta^1: \partial \Gamma \to \mathbb P(\mathbb R^d)$ be a continuous, $\rho(\Gamma)$-equivariant map.
    Since the action of $\Gamma$ on its boundary $\partial \Gamma$ has dense orbits, it suffices to show that it agrees with $\xi^1_\rho$ on at least one boundary point.

    Suppose for the shake of contradiction that $\eta^1$ does not coincide with $ \xi^1$ and let $z \in \partial \Gamma, y \in \partial \Gamma \backslash \{z\}$.
    Then for any $x \in \partial \Gamma \backslash \{y \}$ we may find some quasi-geodesic $\{\gamma_n\}_n$ such that $\gamma_n \to x, \gamma_{-n} \to y$ as $n \to \infty$.
    Then since $z \neq y$ we know that $\gamma_n z \to z$ as $n \to \infty$ and continuity of $\eta^1$ implies that $\eta^1(\gamma_n z) \to \eta^1(z)$.
    But equivariance of $\eta^1$ and the fact that $\xi^1$ is dynamics-preserving implies that $\eta(\gamma_n z) = \rho(\gamma_n) \eta(z) \to \xi^1(x)$,
    unless $\eta^1(z) \in \xi^{d-1}(y)$.
    But if in fact $\eta^1(z) \not \in \xi^{d-1}(y)$, then the limits would agree, i.e.\ $\eta^1(x) = \xi^1(x)$ which is a contradiction.
    Thus we have that $\eta^1(z) \in \xi^{d-1}(y)$ and since $y$ was an arbitrary points of $\partial \Gamma \backslash \{y\}$, we have that
    \[
    \eta^1(z) \in \bigcap_{y \in \partial \Gamma \backslash \{z\}} \xi^{d-1}(y) \subseteq 
    \bigcap_{y \in \partial \Gamma \backslash \Gamma \cdot z} \xi^{d-1}(y).
    \] 
    In particular, the set appearing on the right hand side is $\rho$-equivariant, non-empty proper subset of $\mathbb R^d$, which contradicts the strong irreducibility assumption of $\rho$.
\end{proof}

Given the lemma above, it suffices to find an $\SU(2,1)$-equivariant map from $\partial \Gamma$ to $\mathbb P(\mathbb R^d)$.
\begin{proposition}\label{prop:counterexample_limit_map}
    The projective part of the limit map of $\eta$ is given by
    \[
        \xi_\eta: \partial \Gamma = \SU(2,1)/P_0 \to \mathbb P(\mathfrak{su}(2,1)), \quad \xi_\eta(g P_0) = \mathbb R \Ad_{\gamma} x_0.
    \]
    where  $P_0 = \mathrm{St}_{\SU(2,1)} [1:0:0]$ and 
    \[
    x_0 = i \begin{pmatrix} 0 & 0 & 1 \\ 0 & 0 & 0 \\ 0 & 0 & 0 \end{pmatrix} \in \su(2,1).
    \]
    Its derivative satisfies: 
    \[
    d_x \xi(T_x \SU(2,1)/P_0) = \pi(\ad_{\xi^1(x)} \su(2,1))
    \]
    where $\pi : \hom (\xi^1(x), \su(2,1)) \to \hom(\xi^1(x), \su(2,1)/ \xi^1(x))$ is the canonical projection.
\end{proposition}
\begin{proof}
Since by \cref{lem:equivariance_uniqueness} the limit map is the unique continuous $\rho$-equivariant from the boundary of $\Gamma$ to the projective space, it suffices to show that there exists an $\eta$-equivariant map $\xi^1:\SU(2,1)/P_0 \to \mathbb P(\mathfrak{su}(2,1))$, since it will then restrict to the limit map on $\partial \Gamma$.

We consider the parabolic subgroup $P_0 = \mathrm{St}_{\SU(2,1)} [1:0:0]$ of $\SU(2,1)$.
Then its Lie algebra is given by:
\[
\mathfrak p_0 = \mathrm{St}_{\su(2,1)}[1:0:0] = \left\{ \begin{pmatrix} u - is & a & i t \\ 0 & 2is & - \bar a \\ 0 & 0 & -u - is \end{pmatrix} : a \in \mathbb C, u,s,t \in \mathbb R \right\}.
\]
Since for $\mathbb R x \in \mathbb P(\su(2,1))$ we have that $P_0$ fixes $\mathbb R x$ if and only if $\mathfrak p_0$ fixes $\mathbb R x$.
But a quick calculation shows that the only element of $\su(2,1)$ fixed by $\mathfrak p_0$ is $x_0$.

For the calculation of the image of the differential at the identity coset $P$, we differentiate the commutative diagram:

\begin{tikzcd}
    \SU(2,1) \arrow[r, "\Ad_\cdot x_0"] \arrow[d] & \su(2,1) \arrow[d] \\
    \SU(2,1)/P_0 \arrow[r, "\xi^1"] & \mathbb P(\su(2,1))
\end{tikzcd}
to get
\begin{tikzcd}
    \su(2,1) \arrow[r, "\ad_\cdot x_0"] \arrow[d] & \su(2,1) \arrow[d, "\pi"] \\
    \su(2,1)/\mathfrak p_0 \arrow[r, "\d_P \xi^1"] & T_{\xi^1(P)} \mathbb P(\su(2,1))
\end{tikzcd}

In the general case we use the equivariance of the limit map
\begin{align*}
    \d_{gP} \xi^1 (T_{gP} \SU(2,1)/P_0) &=
    d_{gP} \xi^1 d_P g (T_P \SU(2,1)/P_0) =
    d_{\xi^1(P)} g d_P \xi^1 (T_P \SU(2,1)/P_0) =\\
    &= d_{\xi^1(P)} g \pi(\ad_{\xi^1(P)} \su(2,1)) =\\
    &= \pi(Ad_g(\ad_{\xi^1(P)} \su(2,1))) = \pi(\ad_{Ad_g \xi^1(P)} \su(2,1))=\\
    &= \pi(\ad_{\xi^1(gP)} \su(2,1)).
\end{align*}
\end{proof}

Recall that all parabolic subgroups of $\SU(2,1)$ are conjugate to each other, so we have the following identification:
\begin{align*}
    \SU(2,1)/{P_0} &\leftrightarrow \{ \text{ Parabolic subgroups of } \SU(2,1) \} &\leftrightarrow \{ \text{ Parabolic subalgebras of } \su(2,1) \} \\
    g P_0 &\leftrightarrow g P_0 g^{-1} &\leftrightarrow \Ad_g(\mathfrak p_0)    
\end{align*}

\begin{lemma}\label{lem:adjoint_parabolic}
    Let $\mathfrak p, \mathfrak p' \leq \su(2,1)$ be two distinct parabolic subalgebras.
    Then there exists some $g \in \SU(2,1)$ such that $\Ad_g(\mathfrak p) = \mathfrak p_0$ and $\Ad_g \mathfrak p' = \mathfrak p_0^t$.
\end{lemma}

The following proposition implies that the falsehood of the lemma in the beginning of this chapter.
\begin{proposition}
    Let $\Gamma \leq \SU(2,1)$ be a uniform lattice and $\eta: \Gamma \to \SL(\mathfrak{su}(2,1))$ be the restriction of the adjoint representation.
    Then
    \begin{enumerate}[label=(\roman*)]
        \item $\eta$ is strongly irreducible,
        \item $\eta$ is projective Anosov
        \item $\eta$ admits a measurable $\eta$-equivariant section: 
        \begin{align*}
            \zeta: \partial \Gamma &\to \mathcal F_{\{1, 4 \}}(\mathfrak{su}(2,1)) \simeq \mathcal P\\
            x &\mapsto (\xi^1(x), T_{\xi^1(x)} \xi^1(\partial \Gamma)) \simeq (\xi^1(x), (\d_{\xi^1(x)}p)^{-1}(T_{\xi^1(x)}\xi^1(\partial \Gamma))\xi^1(x)).
        \end{align*}
        where $d_{\xi^1(x)}p: \hom( \xi^1(x), \su(2,1)) \to \hom(\xi^1(x), \su(2,1)/ \xi^1(x))$ is the canonical projection.
        \item For all $x, y \in \partial \Gamma: \zeta(x)^4 \cap \zeta(y)^4 \neq 0$.
        \item For any $y_0 \in \SU(2,1)/P_0$ and 
        $W_0 \in \mathcal G_7(\mathbb R^4)$ that contains $\zeta(y_0)^4$, we have that 
        $\Ann(\zeta(y_0)^4, W_0) \supseteq \zeta(\SU(2,1)/P_0)$ and is in particular of full $\mu$-measure, for any $(\eta(\Gamma), \phi)$-Patterson-Sullivan measure $\mu$ supported over
        $\zeta(\partial \Gamma)$.
    \end{enumerate}
\end{proposition}
\begin{proof}
\begin{enumerate}[label=(\roman*)]
\item Follows from the fact that $\SU(2,1)$ is a simple Lie group.
\item Shown in \cref{prop:counterexample_anosov}.
\item Follows from the fact that $\xi^1$ is $\SU(2,1)$-equivariant and the equivariant identification of $\mathcal F_{\{1, 4 \}}(\mathfrak{su}(2,1)) \simeq \mathcal P$.
\item Letting $g\in \SU(2,1)$ be as in \cref{lem:adjoint_parabolic}, we have that $\Ad_g(\mathfrak p_0) = \mathfrak p$ and $\Ad_g(\mathfrak p_0^t) = \mathfrak p'$.
Thus $\zeta(x)^4 \cap \zeta(y)^4 \neq \emptyset$ if and only if
\begin{align*}
    \emptyset \neq Ad_g (\zeta(x)^4 \cap \zeta(y)^4) &=
    \Ad_g \zeta(x)^4 \cap \Ad_g \zeta(y)^4 = 
    \zeta(g x)^4 \cap \zeta(g y)^4 =
    \zeta(\mathfrak p_0)^4 \cap \zeta(\mathfrak p_0^t)^4=\\
    &= \pi \left( \mathbb R \begin{pmatrix} 1 & 0 & 0 \\ 0 & 0 & 0 \\ 0 & 0 & -1 \end{pmatrix}\right).
\end{align*}
For the last equality, we use \cref{prop:counterexample_limit_map} and the fact that $\mathfrak p_0^t = \Ad_g \mathfrak p_0$ for 
$$g = \begin{pmatrix} 0 & 0 & 1 \\ 0 & 1 & 0 \\ 1 & 0 & 0 \end{pmatrix},$$
to conclude that
\[
\zeta(\mathfrak p_0) = 
\zeta(P_0) = 
\pi\left(
\left\{
\begin{pmatrix} u & a & it \\ 0 & 0 & - \bar a \\ 0 & 0 & -u \end{pmatrix} :
a \in \mathbb C, u, t \in \mathbb R
\right\}
\right),
\]
and
\[
\zeta(\mathfrak p_0^t) = 
\zeta(g P_0) = \Ad_g \zeta(P_0) =
\pi\left(
\left\{
\begin{pmatrix} u & 0 & 0 \\ a & 0 & 0 \\ i t & - \bar a & 0 \end{pmatrix} :
a \in \mathbb C, u, t \in \mathbb R
\right\}
\right).
\]
\end{enumerate}
\end{proof}

\printbibliography
\end{document}